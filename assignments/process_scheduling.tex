\documentclass[12pt]{sftex/sftex}

\title{Implementação de \emph{lottery scheduling} para o Nanvix}
\author{Gustavo F. Guimarães, Gustavo Zambonin, Marcello Klingelfus}
\email{\{gustavo.g,gustavo.zambonin,marcello.klingelfus\}@grad.ufsc.br}
\src{https://github.com/zambonin/nanvix}
\uniclass{Sistemas Operacionais I}
\classcode{UFSC-INE5412}

\begin{document}

\maketitle

\section{Descrição do algoritmo}

Criado com o objetivo de prover garantia de utilização da CPU para cada
processo, a estratégia de escalonamento escolhida para substituir a alternância
circular presente na versão educacional do Nanvix é chamada de \emph{lottery
scheduling}. Este método foi apresentado na literatura em 1994, e consiste na
determinação de uma certa ``moeda'', chamada de \emph{tickets}, cuja função é
representar a porcentagem relativa de execução do processo em relação a todos
os outros. Após a associação de cada processo com uma certa quantidade de
\emph{tickets}, um sorteio é realizado de modo a selecionar o próximo processo
a ser escalonado. \\

Esta estratégia é abstrata o suficiente para ser implementada independente de
detalhes técnicos relativos ao hardware, e justa de acordo com certa
probabilidade, por conta da quantidade de eventos pseudo-aleatórios que ocorrem
a cada escalonamento individual. Entretanto, é possível perceber um aumento
na acurácia da escolha dos processos com um número crescente de sorteios,
como visto em~\cite[seção 2.2]{Waldspurger:1994:LSF:1267638.1267639}. Nota-se
também que, como a alocação de \emph{tickets} e sorteio de processos são
tecnicamente simples, o algoritmo permite a configuração de valores de
\emph{quantum} mais baixos ao longo da execução do sistema operacional.

\section{Características da implementação}

A simplicidade do \emph{kernel} do Nanvix torna a implementação deste algoritmo
bastante concisa; o código inicial não modifica a estrutura de um processo, e
todas as computações relativas aos \emph{tickets} e sorteio são feitas na
função \texttt{yield}, em \texttt{src/kernel/pm/sched.c}. A quantidade de
\emph{tickets} relativa à cada processo é fortemente baseada em sua prioridade
atual, e também leva em conta a chamada de sistema \texttt{nice}. A semente
do gerador de números aleatórios é obtida a partir do tempo do sistema. \\

Porém, observou-se um grande número de cálculos realizados pelos laços internos
à função (cerca de $2n$ execuções de macroinstruções, não otimizadas pelo
compilador, onde $n$ é o tamanho da tabela de processos). Assim, uma solução
otimizada foi elaborada: cada estrutura de processo tem um membro responsável
por guardar seu número de \emph{tickets}, e o gerenciador de processos
(\texttt{src/kernel/pm/pm.c}) guarda um contador total de \emph{tickets}. Estas
variáveis são modificadas pelas funções internas às chamadas de sistema de
criação e destruição de processos (\texttt{fork} e \texttt{bury},
respectivamente). Dessa maneira, a sobrecarga do cálculo de macroinstruções é
diminuída pela metade, e existe uma pequena melhoria no desempenho do teste
focado no problema da inversão de prioridades. \\

Algumas estratégias de gerenciamento de recursos modulares apresentadas em
~\cite[cap. 3]{Waldspurger:1994:LSF:1267638.1267639} não foram implementadas por
conta dos seguintes fatores: no caso da transferência de \emph{tickets} entre
processos, o caso de uso que mais se beneficiaria com este recurso, conversação
entre clientes e servidores, não se aplica no sistema operacional pois este não
implementa comunicação entre máquinas; como a distribuição de processos em um
sistema operacional de propósito geral é heterogênea, implementar a inflação de
\emph{tickets} seria arriscado, pois um processo pode tomar conta do
processador por um período elevado ao dar a si mesmo mais \emph{tickets} por
não ``confiar'' em outros; a criação de uma ``submoeda'' para que cada usuário
classifique seus processos à sua maneira está fora do escopo, visto que o
Nanvix admite apenas um usuário; por fim, a remoção da lógica envolvendo o
membro da estrutura de processo \texttt{counter}, que lida com o caso de um
processo não utilizar seu \emph{quantum} completamente, faz a compensação de
\emph{tickets} não se mostrar necessária neste caso.

\bibliography{process_scheduling}
\bibliographystyle{plain}

\end{document}
