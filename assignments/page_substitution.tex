\documentclass[12pt]{article}

\usepackage[brazil]{babel}
\usepackage[T1]{fontenc}
\usepackage[a4paper, margin=1.5cm]{geometry}
\usepackage[colorlinks, urlcolor=blue, citecolor=red]{hyperref}
\usepackage[utf8]{inputenc}

\title{\textbf{Implementação do algoritmo de envelhecimento
  de páginas para o Nanvix}}
\author{
  Gustavo F. Guimarães, Gustavo Zambonin, Marcello Klingelfus\thanks{
    \texttt{\{gustavo.g,gustavo.zambonin,marcello.klingelfus\}@grad.ufsc.br}
    \hfill \texttt{\href{https://github.com/zambonin/nanvix}{src/}}
  } \\
  \small{Sistemas Operacionais I (UFSC-INE5412)}
}
\date{}

\setlength{\parindent}{0pt}

\begin{document}

\maketitle

\section{Descrição do algoritmo}

A associação de um contador a cada página, relacionado ao bit de referência
(bit R) e modificado a cada interrupção de relógio, é uma maneira razoável de
verificar quais páginas foram referenciadas mais frequentemente; desse modo,
páginas pouco utilizadas serão corretamente removidas da tabela no evento de
uma falta de página. É necessário implementar esse processo de atualização dos
contadores por software, visto que suporte de hardware para este recurso é
esparso. \\

A estratégia chamada de algoritmo do envelhecimento (\emph{aging}) é baseada
neste comportamento, utilizando uma construção específica para os contadores: a
cada $n$ interrupções de relógio (visto que o processo de atualização é
custoso, limita-se seu acionamento), o conteúdo de tais variáveis passa por um
deslocamento aritmético à direita, efetivamente resultando em uma divisão por
2, e o bit R é adicionado ao bit mais significativo, assim representando a
página por um contador ``alto'', consequentemente simbolizando que não deve ser
removida em breve.

\section{Características da implementação}

Inicialmente, é necessário filtrar o número de vezes em que o algoritmo de
envelhecimento é executado. Utilizando-se do contador principal de \emph{ticks}
do sistema, um número $k$ impõe a frequência em que o evento da modificação de
contadores acontece. Para diminuir o preço da verificação de módulo,
$k = 2^n - 1$ pode ser utilizado; assim, apenas uma instrução de máquina é
necessária. Tal número $n$ foi escolhido por inspeção; observa-se que um valor
elevado mascara grande parte das referências às paginas, e um valor pequeno
gera grande ônus de processamento. Por fim, caso o resultado do módulo seja
nulo, a função \texttt{page\_aging} é chamada. \\

Declarada em \texttt{include/nanvix/mm.h} e definida em
\texttt{src/kernel/mm/paging.c}, as entradas da tabela de páginas são
modificadas iterativamente; caso o processo atual seja dono de uma moldura de
página, esta é então traduzida para uma página da tabela, que contém o bit R
necessário para a execução do algoritmo. Os deslocamentos necessários são
realizados e o bit R é zerado, para assegurar que uma página deve ser
referenciada entre os períodos da atualização de contadores. A função
\texttt{allocf} é levemente modificada para que, quando uma página seja
alocada, seu contador inicie com o bit mais significativo ativado, para
demonstrar que foi referenciada.

\end{document}
