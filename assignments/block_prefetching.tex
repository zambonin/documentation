\documentclass[12pt]{article}

\usepackage[brazil]{babel}
\usepackage[T1]{fontenc}
\usepackage[a4paper, margin=1.5cm]{geometry}
\usepackage[colorlinks, urlcolor=blue, citecolor=red]{hyperref}
\usepackage[utf8]{inputenc}

\title{\textbf{Leitura assíncrona de blocos de dados no Nanvix}}
\author{
  Gustavo F. Guimarães, Gustavo Zambonin, Marcello Klingelfus\thanks{
    \texttt{\{gustavo.g,gustavo.zambonin,marcello.klingelfus\}@grad.ufsc.br}
    \hfill \texttt{\href{https://github.com/zambonin/nanvix}{src/}}
  } \\
  \small{Sistemas Operacionais I (UFSC-INE5412)}
}
\date{}

\setlength{\parindent}{0pt}

\begin{document}

\maketitle

\section{Descrição da técnica}

Em um computador, a hierarquia de memória é geralmente organizada considerando
a velocidade de acesso de dados dentro do hardware. O armazenamento secundário,
em sua maioria discos magnéticos, está no nível mais baixo --- longe do
processador e, por conta de seu funcionamento essencialmente mecânico, o mais
lento dos dispositivos de armazenamento de dados. Por conta dessa
característica, é necessário que acessos ao disco sejam reduzidos ou
aproveitados ao máximo, a fim de manter a responsividade do sistema. \\

Para tal, a técnica de \emph{prefetching} pode ser utilizada: consiste da
leitura antecipada de dados do disco, de modo assíncrono, para uma estrutura
localizada junto ao sistema operacional, chamada de \emph{buffer cache}. É
possível observar que, em casos de acesso sequencial a um arquivo, blocos
adjacentes ao desejado serão lidos e assim menos faltas serão geradas na cache
supracitada, por exemplo.

\section{Características da implementação}

De acordo com a descrição da técnica, é necessário que blocos adjacentes
válidos sejam lidos. Assim, a função \texttt{src/kernel/fs/file.c:file\_read}
foi modificada de modo a iterar sobre novos blocos (cuja quantidade é
configurada a partir de \texttt{include/fs/minix.h:BLOCK\_AHEAD}) e, caso estes
contenham dados, sejam lidos de modo assíncrono por
\texttt{src/kernel/fs/buffer.c:bread}. Para que não fosse necessário modificar
todas as chamadas deste, uma nova função \texttt{abread} foi criada com a
lõgica considerando um novo argumento, decidindo se a leitura será assíncrona,
e a função antiga funciona como um \emph{wrapper}, preservando assim a
interface. \\

Na nova função, é necessário considerar a atualização de um atributo
do buffer, mostrando se a leitura será síncrona
(\texttt{src/kernel/fs/fs.h:BUFFER\_SYNC}). Se este for o caso, o fluxo de
código continua o mesmo; de modo contrário, o comportamento do escalonador de
disco será diferenciado por conta da mudança em
\texttt{src/kernel/dev/ata/ata.c:ata\_readblk}, que leva em conta a \emph{flag}
descrita acima para modificar o tipo de requisição
(\texttt{src/kernel/dev/ata/ata.c:REQ\_SYNC}).

\end{document}
